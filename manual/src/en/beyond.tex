% !TEX encoding   = UTF8
% !TEX root       = manual.tex
% !TEX spellcheck = en_GB

\chapter{Beyond this manual}

In this manual, the authors tried to give an overview over {\Tw} and a concise introduction to get you started. {\Tw} is constantly evolving and improving, however, so the information presented here will never be complete.

Additional, frequently updated documents are posted in the wiki hosted by Google Code at \url{http://code.google.com/p/texworks/w/list}. Particularly noteworthy are the following pages:
\begin{description}
\item[SpellingDictionaries] describes how to obtain and install dictionaries for the spell-checker on various systems. \url{http://code.google.com/p/texworks/wiki/SpellingDictionaries}
\item[TipsAndTricks] provides a compilation of useful things to know at a glance, such as the \verb|% !TEX root| construct. \url{http://code.google.com/p/texworks/wiki/TipsAndTricks}
\item[AdvancedTypesettingTools] lists the configurations for several typesetting tools that are not included in {\Tw} by default, such as latexmk or the dvips workflows. \url{http://code.google.com/p/texworks/wiki/AdvancedTypesettingTools}
\end{description}

If you run into problems with {\Tw}, it is advisable to browse the mailing list archives accessible via \url{http://tug.org/pipermail/texworks/}. If you use {\Tw} regularly or are interested in learning about problems and solutions when using it for some other reason, you can also consider subscribing to the list at \url{http://tug.org/mailman/listinfo/texworks} to stay up-to-date. For the occasional post to the mailing list, you can also use the \menu{Help}\submenu\menu{Email to mailing list} menu item. Please make sure you replace the default subject by something describing your issue and to include all information that might help resolving it. That way, you are much more likely to get many helpful replies.

If you find a bug in {\Tw} or want to suggest a new feature you would like to see in a future version, you should have a look at the issue list at Google Code (\url{http://code.google.com/p/texworks/issues/list}). Before posting a new item, please make sure that a similar report or request is not already on the list and that the issue list is indeed the right place, though. If in doubt, please ask on the mailing list first.

Happy {\TeX}ing!